%% Файл стиля для оформления дипломной работы

\documentclass[14pt,a4paper,oneside]{extarticle}

\usepackage[russian]{babel}	% Использование русского языка (работает расстановка переносов)
\usepackage[utf8]{inputenc}	% Кодировка utf8
\usepackage[T2A]{fontenc}	% Шрифт русский, Times New Roman
\usepackage{cmap} 		% Улучшенный поиск русских слов в полученном pdf-файле

\usepackage{indentfirst} 	% Включить отступ у первого абзаца

\usepackage{setspace}		% Полуторный интервал
\onehalfspacing

\usepackage{fancyhdr}		% Нумерация страниц в нижнем правом углу
\fancypagestyle{rightbottom} {
\renewcommand{\headrulewidth}{0pt}
	\fancyhead{}
	\fancyfoot{}
	\fancyfoot[R]{\thepage}
}
\pagestyle{rightbottom}

\renewcommand\contentsname{Содержание} % Переименовываем заголовок Содержания

\usepackage[a4paper, 			% Отступы с краев листа
	left=3.5cm, right=1.5cm, 
	top=2cm, bottom=2cm, 
	footskip=1.25cm]{geometry} 
	
% Заголовки разделов без нумерации
\newcommand{\anonsection}[1]{\section*{#1}\addcontentsline{toc}{section}{#1}}
\newcommand{\anonsubsection}[1]{\subsection*{#1}\addcontentsline{toc}{subsection}{#1}}
\newcommand{\anonsubsubsection}[1]{\subsubsection*{#1}\addcontentsline{toc}{subsubsection}{#1}}

% Точка в конце номаре раздела
\renewcommand{\theequation}{\thesection.\arabic{equation}}
\renewcommand{\thesubsubsection}{\arabic{section}.\arabic{subsection}.\arabic{subsubsection}.}
\renewcommand{\thesubsection}{\arabic{section}.\arabic{subsection}.}
\renewcommand{\thesection}{\arabic{section}.}

\setcounter{tocdepth}{3}	% Включать в оглавление заголовки до 3-го уровня включительно

\usepackage{graphicx} 		% Использование изображений
\graphicspath{{../diagrams/}}	% Путь к каталогу с изображениями

\newcommand{\pic}[1]{\ref{#1}} 	% Ссылка на изображение
\newcommand{\tab}[1]{\ref{#1}} 	% Ссылка на таблицу

\usepackage[tableposition=top]{caption}
\usepackage{subcaption}
\DeclareCaptionLabelFormat{gostfigure}{\textbf{Рис. #2.}}	% Подпись к рисунку
\DeclareCaptionLabelFormat{gosttable}{\textbf{Табл. #2.}}	% Подпись к таблице
\DeclareCaptionLabelSeparator{gost}{~ ~}
\captionsetup{labelsep=gost}
\captionsetup[figure]{labelformat=gostfigure}
\captionsetup[table]{labelformat=gosttable}
\renewcommand{\thesubfigure}{\asbuk{subfigure}}

\usepackage{multirow}		% Возможность объединения нескольких ячеек таблицы в одну

\makeatletter				% Изменение оформления нумерации списка литературы
\renewcommand{\@biblabel}[1]{#1.\hfil}
\makeatother
\bibliographystyle{utf8gost705u}	% Cтилевой файл для оформления по ГОСТу

\begin{document}
\thispagestyle{empty}
\begin{singlespacing}

\begin{center}
{\small МИНИСТЕРСТВО ОБРАЗОВАНИЯ И НАУКИ РОССИЙСКОЙ ФЕДЕРАЦИИ 	\\*
Федеральное государственное бюджетное образовательное учреждение \\*
высшего профессионального образования}							\\*
\textbf{«Южно-Уральский государственный университет» 			\\*
(национальный исследовательский университет) 					\\*
{\small Факультет Вычислительной математики и информатики 		\\*
Кафедра системного программирования}}
\end{center}

\vspace{60pt}

\begin{center}
\textbf{КУРСОВОЙ ПРОЕКТ} \\*
бакалавра направления 010400.62 "Информационные технологии" \\*
по дисциплине "Технологии анализа данных"
\end{center}

\vspace{24pt}

\begin{center}
\textbf{Разработка приложения интеллектуального анализа данных \\*
на платформе СУБД MySQL}
\end{center}

\vspace{24pt}

\begin{tabular}{p{0.5\linewidth}p{0.5\linewidth}}
&
Выполнил:			 		\par
студент группы ВМИ-456		\par
Е.М. Лукичева				\par
~							\par
~							\par
Проверил:					\par
М.Л. Цымблер					\par
Оценка: \makebox[1.0in]{\hrulefill}\par
Подпись:	 \makebox[1.0in]{\hrulefill}\par
Дата: \makebox[1.0in]{\hrulefill}\par
\\
\end{tabular}

\vfill

\begin{center}
Челябинск-2014
\end{center}

\end{singlespacing}

\section{Задание}
\subsection{Предметная область}
Компания занимается сборочным  производством и  продажей сложных устройств из деталей, закупаемых у поставщиков. Компания имеет два филиала, которые географически удалены друг от друг и имеютотличающуюся информационную структуру.\par
Аналитик компании выполняет подготовку  различных  оперативных  и аналитических  отчетов, целью которых  является увеличение эффективности деятельности компании в целом.\par
Необходимо  разработать программную систему анализа бизнес-данных для аналитика компании, которая выполняет следующие основные функции: интеграция  данных  из  филиалов  в  хранилище данных  компании, подготовка оперативных и аналитических отчетов.\par
Далее приведено описание сущностей предметной области. \par
При описании сущностей используются обозначения, указанные в \tab{semantic}.
\begin{table}[h]
	\caption{\space Обозначения атрибутов сущностей}
	\label{semantic}
	\begin{tabular}{|p{5cm}|p{10cm}|}
		\hline
		\textbf{Обозначение} & \textbf{Семантика}\\
		\hline
		* & Атрибут является первичным ключом сущности\\
		\hline
		\textasciicircum Сущность.Атрибут & Атрибут является внешним ключом и ссылается на указанный атрибут указанной сущности\\
		\hline
	\end{tabular}
\end{table}

В предметной области выделены сущности Поставщик, Деталь и Поставка. Описание сущности Поставщик представлено в табл. \tab{supplier}
\begin{table}[h]
	\caption{\space Атрибуты сущности Поставщик (S)}
	\label{supplier}
	\begin{tabular}{|p{0.4cm}|p{2.5cm}|p{1.5cm}|p{7cm}|p{3.5cm}|}
		\hline
		\textbf{№} & \textbf{Атрибут} & \textbf{Ключ} & \textbf{Семантика} & \textbf{Тип данных} \\
		\hline
		1. & SID & * & Уникальный код поставщика & INT \\
		\hline
		2. & SName &  & Имя поставщика & CHAR(20) \\
		\hline
		3. & SCity & & Город поставщика & CHAR(20) \\
		\hline
		4. & Address & & Почтовый адрес поставщика & CHAR(20) \\
		\hline
		5. & Risk & & Риск сотрудничества с поставщиком (низкий, средний, высокий) & (1, 2, 3) \\
		\hline
	\end{tabular}
\end{table}

Атрибуты сущности Деталь представлены в табл. \tab{part}.
\begin{table}[h]
	\caption{\space Атрибуты сущности Деталь (P)}
	\label{part}
	\begin{tabular}{|p{0.4cm}|p{2.5cm}|p{1.5cm}|p{7cm}|p{3.5cm}|}
		\hline
		\textbf{№} & \textbf{Атрибут} & \textbf{Ключ} & \textbf{Семантика} & \textbf{Тип данных} \\
		\hline
		1. & PID & * & Уникальный код детали & INT \\
		\hline
		2. & PName &  & Имя детали & CHAR(20) \\
		\hline
		3. & HTP & & Является ли деталь продуктом высоких технологий (High Technology Product) & BOOL \\
		\hline
		4. & Weight & & Вес детали в килограммах & FLOAT \\
		\hline
	\end{tabular}
\end{table}

Описание атрибутов сущности  Поставка представлено в табл. \tab{relation}.
\begin{table}[h]
	\caption{\space Атрибуты сущности Поставка (SP)}
	\label{relation}
	\begin{tabular}{|p{0.4cm}|p{2.5cm}|p{1.5cm}|p{7cm}|p{3.5cm}|}
		\hline
		\textbf{№} & \textbf{Атрибут} & \textbf{Ключ} & \textbf{Семантика} & \textbf{Тип данных} \\
		\hline
		1. & SPID & * & Уникальный код поставки & INT \\
		\hline
		2. & SID & \textasciicircum S.SID & Уникальный код поставщика & INT \\
		\hline
		3. & PID & \textasciicircum P.PID & Уникальный код детали & INT \\
		\hline
		4. & Qty & & Количество деталей в поставке & INT \\
		\hline
		5. & Price & & Цена за 1 шт. & FLOAT \\
		\hline
		6. & OrderDate & & Дата заказа поставки & DATE \\
		\hline
		7. & Period & & Срок доставки в днях & INT \\
		\hline
		8. & ShipDate & & Фактическая дата доставки & DATE \\
		\hline
	\end{tabular}
\end{table}

Данные  в  каждом из  филиалов  должны  подчиняться ограничениям целостности, перечисленным в табл. \tab{integrity}.Тем не менее,в филиалах не всегда осу-ществляется проверка целостности вводимых данных, вследствие чего в дан-ных возможны ошибки.\par
\begin{table}[h]
	\caption{\space Ограничения целостности}
	\label{integrity}
	\begin{tabular}{|p{4cm}|p{8cm}|}
		\hline
		\textbf{Сущность} & \textbf{Ограничение целостности} \\
		\hline
		\multirow{4}{*}{S} 	& SName NOT NULL \\
					& SCity NOT NULL \\
					& Address NOT NULL \\
					& (SName, Address, SCity) UNIQUE \\
		\hline
		\multirow{3}{*}{P}	& PName NOT NULL\\
					& Weight > 0\\
					& HTP in (0, 1) \\
					& (PName, Weight) UNIQUE \\
		\hline
		\multirow{5}{*}{SP}	& OrderDate NOT NULL \\
					& Qty > 0 \\
					& Price > 0 \\
					& Period >= 0 \\
					& OrderDate <= ShipDate \\
		\hline
		S, P, SP		& OrderDate NOT NULL \\
		\hline
	\end{tabular}
\end{table}
В филиале №1 для обработки данных используется СУБД MySQL. База данных представляет собой совокупность реляционных таблиц S, P, SP, структура которых описана в разделе 1.1.\par
В филиале №2 для обработки данных используется СУБД SQLite. База данных представляет собой совокупность реляционных таблиц S, P, SP, структура которых описана в разделе 1.1.\par
\subsection{Функции программной системы}
\section{Проектирование}
\subsection{Система хранилища данных}
\subsection{ETL}
\subsection{Функции OLAP}
\subsection{Функции интеллектуального анализа данных}
\subsection{Интерфейс пользователя}
\section{Реализация}

\end{document}