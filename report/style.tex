%% Файл стиля для оформления дипломной работы

\documentclass[14pt,a4paper,oneside]{extarticle}

\usepackage[russian]{babel}	% Использование русского языка (работает расстановка переносов)
\usepackage[utf8]{inputenc}	% Кодировка utf8
\usepackage[T2A]{fontenc}	% Шрифт русский, Times New Roman
\usepackage{cmap} 		% Улучшенный поиск русских слов в полученном pdf-файле

\usepackage{indentfirst} 	% Включить отступ у первого абзаца

\usepackage{setspace}		% Полуторный интервал
\onehalfspacing

\usepackage{fancyhdr}		% Нумерация страниц в нижнем правом углу
\fancypagestyle{rightbottom} {
\renewcommand{\headrulewidth}{0pt}
	\fancyhead{}
	\fancyfoot{}
	\fancyfoot[R]{\thepage}
}
\pagestyle{rightbottom}

\renewcommand\contentsname{Содержание} % Переименовываем заголовок Содержания

\usepackage[a4paper, 			% Отступы с краев листа
	left=3.5cm, right=1.5cm, 
	top=2cm, bottom=2cm, 
	footskip=1.25cm]{geometry} 
	
% Заголовки разделов без нумерации
\newcommand{\anonsection}[1]{\section*{#1}\addcontentsline{toc}{section}{#1}}
\newcommand{\anonsubsection}[1]{\subsection*{#1}\addcontentsline{toc}{subsection}{#1}}
\newcommand{\anonsubsubsection}[1]{\subsubsection*{#1}\addcontentsline{toc}{subsubsection}{#1}}

% Точка в конце номаре раздела
\renewcommand{\theequation}{\thesection.\arabic{equation}}
\renewcommand{\thesubsubsection}{\arabic{section}.\arabic{subsection}.\arabic{subsubsection}.}
\renewcommand{\thesubsection}{\arabic{section}.\arabic{subsection}.}
\renewcommand{\thesection}{\arabic{section}.}

\setcounter{tocdepth}{3}	% Включать в оглавление заголовки до 3-го уровня включительно

\usepackage{graphicx} 		% Использование изображений
\graphicspath{{img/}}		% Путь к каталогу с изображениями

\newcommand{\pic}[1]{\ref{#1}} 	% Ссылка на изображение
\newcommand{\tab}[1]{\ref{#1}} 	% Ссылка на таблицу

\usepackage[tableposition=top]{caption}
\usepackage{subcaption}
\DeclareCaptionLabelFormat{gostfigure}{\textbf{Рис. #2.}}	% Подпись к рисунку
\DeclareCaptionLabelFormat{gosttable}{\textbf{Табл. #2.}}	% Подпись к таблице
\DeclareCaptionLabelSeparator{gost}{~ ~}
\captionsetup{labelsep=gost}
\captionsetup[figure]{labelformat=gostfigure}
\captionsetup[table]{labelformat=gosttable}
\renewcommand{\thesubfigure}{\asbuk{subfigure}}

\usepackage{multirow}		% Возможность объединения нескольких ячеек таблицы в одну

% Изменение оформления нумерации списка литературы
\makeatletter
\renewcommand{\@biblabel}[1]{#1.\hfil}
\makeatother
\bibliographystyle{utf8gost705u}	% Cтилевой файл для оформления по ГОСТу